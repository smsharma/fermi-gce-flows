\documentclass[prd,aps,10pt,nofootinbib,twocolumn,superscriptaddress,preprintnumbers,balancelastpage,longbibliography]{revtex4-1}

\usepackage{amsmath,amssymb}	
\usepackage{mathtools}
\usepackage{fontawesome}
\usepackage[dvipsnames]{xcolor}
\usepackage{hyperref}
\usepackage{xspace}
\usepackage{fancyhdr}
\usepackage{braket}
\usepackage{graphicx}
\usepackage{siunitx}
\usepackage{blindtext}
\usepackage{nicefrac}
\usepackage{lipsum}

\usepackage{afterpage}
\newcolumntype{L}[1]{>{\raggedright\let\newline\\\arraybackslash\hspace{0pt}}m{#1}}
\newcolumntype{C}[1]{>{\centering\let\newline\\\arraybackslash\hspace{0pt}}m{#1}}
\newcolumntype{R}[1]{>{\raggedleft\let\newline\\\arraybackslash\hspace{0pt}}m{#1}}
\usepackage{longtable}
\setlength{\LTcapwidth}{\textwidth}

\newcommand{\nbicon}{{\color{linkcolor}\faFileCodeO}\xspace}
\newcommand{\nblink}[1]{\href{https://github.com/smsharma/dark-photons-perturbations/blob/apr-2020/notebooks/#1.ipynb}{\nbicon}}
\newcommand{\githubmaster}{\href{https://github.com/smsharma/dark-photons-perturbations/}{\faGithub}\xspace}

\newcommand{\Fermi}{\emph{Fermi}\xspace}
\newcommand{\HEALPix}{\texttt{HEALPix}\xspace}
\newcommand{\dd}{\mathrm{d}}
\newcommand{\vect}[1]{\boldsymbol{\mathbf{#1}}}

\definecolor{deepgreen}{rgb}{0.2,0.8,0.2}
\newcommand{\SM}[1]{{\bf \color{deepgreen}{[SM: #1]}}}

\colorlet{linkcolor}{BrickRed}



\hypersetup{colorlinks=true,
linkcolor=linkcolor,
citecolor=linkcolor,
urlcolor=linkcolor,
,linktocpage=true
,pdfproducer=medialab}

\DeclareSIUnit \h {\ensuremath{\mathit{h}}}
\DeclareSIUnit\electronvolt{e\kern-.05em V}
\DeclareSIUnit\parsec{pc}

\begin{document}

\title{A neural simulation-based inference approach for characterizing \\ the Galactic Center $\gamma$-ray excess}
 
\author{Siddharth Mishra-Sharma}
\email{sm8383@nyu.edu}
\thanks{ORCID: \href{https://orcid.org/0000-0001-9088-7845}{0000-0001-9088-7845}}
\affiliation{Center for Cosmology and Particle Physics, Department of Physics, New York University, New York, NY 10003, USA}

\author{Kyle Cranmer}
\email{kyle.cranmer@nyu.edu}
\thanks{ORCID: \href{https://orcid.org/0000-0002-5769-7094}{0000-0002-5769-7094}}
\affiliation{Center for Cosmology and Particle Physics, Department of Physics, New York University, New York, NY 10003, USA}
\affiliation{Center for Data Science, New York University, 60 Fifth Ave, New York, NY 10011, USA}

\date{\today}

\begin{abstract}
The nature of the \Fermi $\gamma$-ray Galactic Center Excess (GCE) has remained a persistent mystery for over a decade. Although the excess is broadly compatible with emission expected from annihilating dark matter, an explanation in terms of a population of unresolved point sources remains viable. The effort to uncover its origin is hampered in particular by the mischaracterization of the diffuse emission of Galactic origin, which can lead to spurious residuals that make it difficult to robustly differentiate smooth emission, as expected of dark matter origin, from more clumpy emission sources from relatively bright, unresolved point sources. We use neural simulation-based inference methods, in particular conditional density estimation, in order to extract more information from $\gamma$-ray maps of the Galactic Center with the aim of characterizing the contribution of unresolved point sources to the GCE.
\end{abstract}

\maketitle

\section{Introduction}
\label{sec:intro}

Dark matter (DM) represents one of the major unsolved problems in particle physics and cosmology today. The traditional Weakly-Interacting Dark Matter (WIMP) paradigm envisions production of dark matter in the early Universe through freeze-out of dark matter particles weakly coupled to the Standard Model (SM). In this scenario, one of the most promising methods of detecting a dark matter signal would be through an excess of $\gamma$-ray photons from DM-rich regions of the sky, produced through the cascade of SM particles produced as a result of DM annihilation. 

The \Fermi $\gamma$-ray Galactic Center Excess (GCE), first identified over a decade ago using data from the \Fermi Large Area Telescope (LAT), is an excess of photons in the Galactic Center with properties---such as energy spectrum and spatial morphology---roughly compatible with that expected due to annihilation DM. The nature of the GCE remains contentious---although it's properties are similar to that expected from DM annihilation, competing explanations in terms of a population of unresolved astrophysical point sources (PSs) remain viable. Although analyses based on the statistics of photon counts in the Galactic Center have shown the data to prefer a point source origin of the excess, recent analyses have pointed out the potential of unknown systematics in the data---such as the morphology of the diffuse foreground emission---to bias the conclusions of these results. Furthermore, analyses of the spatial morphology of the excess have shown it to be more compatible with that tracing a stellar bulge distribution in the Galactic Center.

Novel analysis techniques that can extract more information from the high-dimensional photon count maps could do a better job of ``hedging'' against unknown systematics in the data compared to traditional analyses based on summary statistics like the 1-point PDF. Machine learning methods in particular have demonstrated promise for general analyses of $\gamma$-ray data and in particular focusing on understanding the nature of the GCE. In this paper, we leverage recent developments in the field of simulation-based inference (SBI) in order to weigh in on the nature of the GCE.

This paper is organized as follows. In Sec.~\ref{sec:analysis} we describe the various components of our analysis method based on simulation-based inference. In Sec.~\ref{sec:simulations} we describe our simulation pipeline and verify our analysis on simulated data. Section~\ref{sec:data} presents results on \Fermi $\gamma$-ray data. In Sec.~\ref{sec:mismodeling} we explore the susceptibility of the analysis to mismodeling of the background and signal templates, and present systematic variations on our analysis in Sec.~\ref{sec:systematics}. We conclude in Sec.~\ref{sec:conclusion}.

\section{Methodology}
\label{sec:analysis}

We begin by describing our analysis methodology, going over, in turn the general principles behind simulation-based inference, posterior estimation using normalizing flows, and learning representative summary statistics from high-dimensional $\gamma$-ray maps with neural networks.

\subsection{Simulation-based inference}

Of central interest in parameter estimation is often the probability distribution of a set of parameters of interest $\theta$ given some data $x$---the posterior distribution $p(\theta\mid x)$. Bayes' theorem can be used to obtain the posterior as $p(\theta\mid x) = p(\theta)\, p(x\mid\theta) / \mathcal Z$, where $p(x\mid\theta)$ is the likelihood and $\mathcal Z$ is the Bayesian evidence. In practice, parameters other than $\theta$---latent variables $z$---are often involved in the data-generation process, and computing the likelihood involves marginalizing over the latent space, $p(x\mid\theta) = \int \dd z\,p(x\mid\theta, z)$. In typical problems of interest, the high dimensionality of the latent space often means that this integral is intractable, necessitating simplifications in statistical treatment as well as theoretical modeling. 

% $\theta \sim p(\theta)$, $x\sim p(x\mid\theta, z)$, $z\sim p(z\mid\theta)$.

Simulation-based inference (SBI) refers to a class of methods for performing inference when the data-generating process does not have a tractable likelihood. In this setting, a model is defined through a simulator as a probabilistic program, often knows as a forward model. Samples $x$ from the simulator then implicitly define a likelihood, $x\sim p(x\mid\theta)$. In the simplest realizations of SBI, samples $x'$ generated from a given prior proposal distribution $p(\theta$) can be compared to a given dataset of interest $x$, with the approximate posterior defined by samples that most closely approximate $x$. Such methods---usually grouped under the umbrella term Approximate Bayesian Computation (ABC)---are not uncommon in astrophysics and cosmology. Nevertheless, they suffer from several downsides. The curse of dimensionality usually necessitates reduction of data to representative lower-dimensional summary statistics $s(x)$, resulting in loss of information. A notion and measure of distance between summaries from the implicit model and those derived from the dataset of interest is necessary, leading to inexact inference. Additionally, the ABC analysis must be performed anew for each new target dataset.

Recently, methods leveraging advancements in machine learning and probabilistic programming have been developed in order to address these issues, enabling new ways of performing inference on complex models defined through simulations. See Ref.~\cite{cranmer2020frontier} for a review of recent developments.

\subsection{Conditional density estimation with normalizing flows}

In this paper, we approximate the joint posterior $p(\theta\mid x)$ through a parameterized distribution $\hat p_\phi(\theta\mid s)$ conditioned on summaries $s=s(x)$ from the samples $x$. This class of simulation-based inference techniques, known as conditional neural density estimation, directly models the posterior distribution given a set of samples drawn from simulator according to some prior proposal distribution $p(\theta)$.

We employ normalizing flows, which provide a general way of constructing flexible probability distributions, for posterior estimation. Specifically, we use Masked Autoregressive Flows (MAFs).

\subsection{Learning summary statistics with neural networks}

The curse of dimensionality makes it computationally prohibitive to condition the density estimation task on the the raw dataset $x$ \emph{i.e.}, the $\gamma$-ray pixel counts map in the region of interest (ROI). Instead, we use a neural network to learn lower-dimensional summary features from the map, $s = s_\varphi(x)$. The \emph{DeepSphere}~\cite{defferrard2020deepsphere,Perraudin:2018rbt} architecture, with a configuration similar to that introduced and employed in Ref.~\cite{List:2020mzd}, is used here and summarized briefly here. For more details on the summary extractor network, see Ref.~\cite{List:2020mzd}.

\subsection{Simulations and datasets}

We use the datasets and templates from Ref.~\cite{rodd_nicholas_safdi_siddharth_2016} (packaged with Ref.~\cite{Mishra-Sharma:2016gis}) to create the simulated maps. The data and templates used correspond to 413 weeks of \emph{Fermi}-LAT Pass 8 data collected between August 4, 2008 and July 7, 2016. The top quarter of photons in the energy range 2--20~GeV by quality of PSF reconstruction (corresponding to PSF3 event type) in the event class \texttt{ULTRACLEANVETO} are used. The recommended quality cuts are applied, corresponding to zenith angle less than 90$^\circ$, \texttt{LAT\_CONFIG} = 1, and \texttt{DATA\_QUAL} $> 0.1$.\footnote{\url{https://fermi.gsfc.nasa.gov/ssc/data/analysis/documentation/Cicerone/Cicerone_Data_Exploration/Data_preparation.html}} The maps are spatially binned using HEALPix~\cite{Gorski:2004by} with \texttt{nside} = 128. Note that here we only use the real \Fermi-LAT counts data to determine the appropriate normalizations for the Poissonian astrophysical background templates.

The simulated data maps are a combination of smooth (\emph{i.e.}, Poissonian) and PS contributions. Each PS population is completely specified by its spatial and source-count distribution. Photon counts from a generated PS population are put down on a map according to the same \emph{Fermi} PSF described in the previous section using the algorithm implemented in the code package \texttt{NPTFit-Sim}~\cite{NPTFit-Sim}.

In addition to the PS emission from subhalos and blazars, we also account for Poissonian astrophysical emission in the simulated maps.  These contributions include: \emph{(i)}~the Galactic diffuse foreground emission, described by \texttt{Model~O} from Ref.~\cite{Buschmann:2020adf} and found in that reference to be formally the best fit of the considered models to \emph{Fermi} data at higher latitudes $|b|>20^\circ$, \emph{(ii)}~isotropic emission, \emph{(iii)}~resolved PSs from the 3FGL catalog, and \emph{(iv)}~emission from the \Fermi bubbles~\cite{Su:2010qj}. The latter three templates are obtained from Ref.~\cite{rodd_nicholas_safdi_siddharth_2016} and the normalization of each template is set to the best-fit coefficient from a Poissonian regression of the templates to the \emph{Fermi} data in the analysis ROI. To get the normalization of the Poissonian isotropic template, we subtract the modeled blazar contribution from the best-fit isotropic Poissonian emission. This ensures that the total IGRB flux is consistent with observations. The final maps are obtained by combining a Poisson-fluctuated realization of the combined astrophysical templates with the subhalo and blazar PS maps. We use the provided PS mask~\cite{rodd_nicholas_safdi_siddharth_2016} to mask resolved 3FGL PSs at 95\% PSF containment.   

\subsection{Optimization and training}

$10^{6}$ training samples are generated. Models are trained with batch size 64 using the \texttt{AdamW} optimizer with initial learning rate $10^{-3}$ and weight decay $10^{-5}$. Cosine annealing is used to decay the learning rate, training for up to 100 epochs with early stopping if the validation loss has not improved after 10 epochs.

\section{Tests on simulated data}
\label{sec:simulations}

We begin by validating our trained model on simulated data. We create simulated datasets using the best-fit astrophysical model obtained from a non-Poissonian fit in our fiducal ROI, and test the ability of our model to infer the presence of either DM-like or PS-like signals on top of this background.

Figure~\ref{fig:sim_sbi_dm} shows results on maps where the GCE consists of purely DM-like emission. The left column shows the middle-68/95\% containment of the point-wise posterior on the source-count distributions of GCE- and disk-correlated point source emission in red and blue, respectively. The middle column shows the posterior on emission of all components in our model. The right- column shows the fraction of DM- and PS-like emission in proportion to the total inferred flux in the ROI. The true underlying quantities from which the data was generated are represented by dotted lines. We see that, in all cases shown, we are successfully able to infer the presence of DM-like emission. Some PS-like emission is inferred in most cases as well, due to a combination of degeneracy with both disk-correlated and DM-like flux. The overall flux of all components corresponds well to their true underlying values.

Figure~\ref{fig:sim_sbi_dm} shows the corresponding results for simulated data containing PS-like emission. We see that PS-like emission is successfully inferred in each case. Furthermore, the model is able to characterize the source-count of PSs through the source count distribution. Some degeneracy with disk-like source is seen.

%
\begin{figure*}
    \centering
    \includegraphics[width=0.95\textwidth]{plots/sim_sbi_dm.pdf}
    \caption{Results on simulated data where the GCE consists of purely DM-like emission.}
    \label{fig:sim_sbi_dm}
\end{figure*}
%

%
\begin{figure*}
    \centering
    \includegraphics[width=0.95\textwidth]{plots/sim_sbi_ps.pdf}
    \caption{Results on simulated data where the GCE consists of purely PS-like emission.}
    \label{fig:sim_sbi_ps}
\end{figure*}
%

\section{Results on \emph{Fermi} data}
\label{sec:data}

We finally apply our formalism to real \Fermi data. The results are shown in Fig.~\ref{fig:fid_data}. The top panel shows results from this work, while the bottom row shows results from an NPTF analysis. No strong preference for either PS-like or DM-like emission is seen, contrary to the NPTF analysis.

Figure~\ref{fig:sig_inj_data} shows the results of injecting a DM signal directly onto the \Fermi data. We see that the injected DM signal is successfully recovered and not misattributed to PS-like emission.

%
\begin{figure*}
    \centering
    \includegraphics[width=0.95\textwidth]{plots/data_fid_sbi.pdf}
    \includegraphics[width=0.95\textwidth]{plots/data_fid_nptf.pdf}
    \caption{Fiducial results on data.}
    \label{fig:fid_data}
\end{figure*}
%

\subsection{Signal injection test}
\label{sec:sig-injection}

%
\begin{figure*}
    \centering
    \includegraphics[width=0.95\textwidth]{plots/data_sig_inj.pdf}
    \caption{Signal injection on data.}
    \label{fig:sig_inj_data}
\end{figure*}
%

\section{Susceptibility to mismodeling}
\label{sec:mismodeling}

\subsection{Foreground mismodeling}
\label{sec:fg-mismodeling}

\subsection{Signal mismodeling}
\label{sec:sig-mismodeling}

\section{Systematic variations on analysis}
\label{sec:systematics}

\begin{itemize}
    \item Different diffuse models (Models A, B, F, p6v11, mismodeling, GP data-driven)
    \item Different ROIs (15, 20, 25)
    \item Varying inner slope of NFW profile
    \item Different disk templates
\end{itemize}

\section{Conclusions}
\label{sec:conclusion}

The code used to obtain in this paper is available at \url{https://github.com/smsharma/sbi-fermi}.

\vspace{.3cm}
%%%%%%%%%%%%%%%

\begin{acknowledgments}

We thank Florian List, Nick Rodd, and Tracy Slatyer for helpful conversations.  
KC is partially supported by NSF awards ACI-1450310, OAC-1836650, and OAC-1841471, the NSF grant PHY-1505463, and the Moore-Sloan Data Science Environment at NYU. 
SM is supported by the NSF CAREER grant PHY-1554858, NSF grants PHY-1620727 and PHY-1915409, and the Simons Foundation. 
This work made use of the NYU IT High Performance Computing resources, services, and staff expertise. 
This research has made use of NASA's Astrophysics Data System. 
This research made use of the \texttt{astropy}~\cite{Price-Whelan:2018hus,Robitaille:2013mpa}, \texttt{dynesty}~\cite{Speagle_2020}, \texttt{IPython}~\cite{PER-GRA:2007}, Jupyter~\cite{Kluyver2016JupyterN}, \texttt{matplotlib}~\cite{Hunter:2007}, \texttt{mlflow}, \texttt{NPTFit}~\cite{Mishra-Sharma:2016gis}, \texttt{NumPy}~\cite{numpy:2011}, \texttt{PyTorch}~\cite{NEURIPS2019_9015}, \texttt{PyTorch Geometric}~\cite{Fey/Lenssen/2019}, \texttt{PyTorch Lightning}~\cite{william_falcon_2020_3828935}, \texttt{seaborn}~\cite{seaborn}, \texttt{pandas}~\cite{pandas:2010}, \texttt{sbi}~\cite{tejero-cantero2020sbi}, \texttt{scikit-learn}~\cite{scikit-learn}, \texttt{SciPy}~\cite{2020SciPy-NMeth}, and \texttt{tqdm}~\cite{da2019tqdm}  software packages. 
\end{acknowledgments}

% \appendix

% \section{Variations on analysis}
% \label{app:variations}

\bibliographystyle{apsrev4-1}
\bibliography{fermi-gce-sbi}

\end{document}
